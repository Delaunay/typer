\chapter{Présentation de l'\udem{}}

    \section{L'université}

    \structpar{Chiffres}
        L'\udem{} est une université située à Montréal au Canada.
        Elle a été créée en 1878.
        Tout les départements de l'\udem{} sont situés sur le même campus sur le Mont Royal.

        Elle est classée première université du Québec\cite{udemnum} et deuxième université du Canada\cite{udemnum}, que ce soit en nombre
        d'étudiants ou pour la recherche.

        \structpar{Chiffres}
        Cette université accueille près de \num{67 000} étudiants et \num{2 700} chercheurs dans 16 facultés et écoles\cite{udemnum}.
        Elle dispose d'un budget de \num{1,05} milliard de dollards canadiens\cite{wikinum} (en 2014).

        L'\udem{} propose des programmes dans différents domaines d'études comme le droit, les sciences, la littérature \dots
        \begin{figure}[!h]
            \centering
            \includegraphics[width=0.8\linewidth]{./img/DSCN0360.JPG}
            \caption{\udem{}}
\label{fig:udem}
        \end{figure}
        \clearpage

        \section{Le département informatique}

        \structpar{Cursus possibles}
        Le département informatique propose plusieurs cursus allant du développement web à la compilation.

        \structpar{LTP}
        Le laboratoire dans lequel j'ai fait mon stage est le Laboratoire de Traitement Parallèle.
        Il est composé de deux professeurs: Stefan Monnier et Marc Feeley.

        Initialement, ce laboratoire était dédié à des travaux sur le parallélisme.
        Cependant, actuellement, le sujet de recherche des professeurs de ce laboratoire est la compilation.
        \begin{figure}[!h]
            \centering
            \includegraphics[width=0.8\linewidth]{./img/DSCN0410.JPG}
            \caption{Laboratoire de Traitement Parallèle}
\label{fig:ltp}
        \end{figure}

        \structpar{Stefan Monnier}
        Mon tuteur de stage était Stefan Monnier. Actuellement, ses sujets de recherche sont les langages de programmation fortement typés et les
        langages sécurisés ainsi que les répercussions de ces langages sur les compilateurs, les \textit{runtimes system} et les systèmes d'exploitations.
        Ces recherches l'ont amené à créer le langage \typer{} qui sert de preuve de concept pour un langage sécurisé et qui fournit des outils
        pour la création de preuves de logiciels.
