\chapter{Le langage \typer{}}

    \section{Présentation du langage}
        \structpar{Description}
        \typer{} est un langage de programmation fonctionnel pur. Il est total, typé statiquement et avec des types dépendants.
        Les macros de \typer{} permettent de faire de la méta-programmation et d'étendre le langage

        Dans la suite du rapport, \say{\typer{}} fera référence au langage \typer{} et \say{le compilateur \typer{}} fera référence au compilateur
        interprétant le langage \typer{}

        \structpar{Inspirations}
        \typer{} est inspiré de \textit{Lisp}, \textit{Coq} et \textit{ML}.
        Il est conceptuellement proche de \textit{Coq} (langage pur, total, typé statiquement et avec des types dépendants). Cependant, le but de \typer{}
        est d'écrire des programmes et non des preuves.
        Le système de macros de \typer{} est inspiré de \textit{Lisp}, il permet d'étendre facilement le langage
        en manipulant le code comme des données. De plus, comme pour \textit{Lisp}, la syntaxe de \typer{} est stratifiée.
        La syntaxe du code est proche de la syntaxe de \textit{ML}. L'inférence de type de \typer{} est un sur-ensemble de l'inférence de type
        \textit{Hindley-Milner}
        \structpar{Spécificités}
        Les macros de \typer{} devraient, à terme, pouvoir être utilisées pour construire des preuves de programmes.

    \section{Motivations}
        \typer{} est un langage dont le but est de fournir des moyens d'écrire du code sûr.
        En effet, les outils qu'il devrait fournir, à terme, facilitent l'écriture de code correct.
        Par exemple, le compilateur de \typer{} peut vérifier qu'il n'y aura pas de récursion infinie.
        De plus, les macros du langage devrait permettre, à terme, d'écrire des preuves de code.

    \section{État de l'implémentation}
        \structpar{Avant}
        Au début de mon stage, les bases du langage étaient implémentées:
        la définition et l'appel de lambda, la création de type ainsi que la création de variables étaient supportées par le compilateur.
        L'inférence de type n'étant pas implémentée, le compilateur requérait le typage explicite des variables et des paramètres de fonction.
        Le compilateur fournissait peu de fonctions et opérations.

        \structpar{Après}
        \`A la fin du stage, une version basique de l'inférence de type était en cours d'intégration: il restait quelques bogues à résoudre avant que
        cette version soit fonctionnelle. En effet, l'intégration de l'unification à système de typage a requis certaines modifications dans
        ce système. Ces modifications ont entrainées des erreurs qui n'ont pas pu être résolues à temps.
