\chapter{Conclusion}

Ce stage m'a permis de découvrir une partie de l'informatique que nous n'avons pas vu en cours:
les mécanismes d'inférence de type et l'interprétation de langage fonctionnel.
Il m'a permis de découvrir le $\lambda{}-calcul$, modèle de calcul utilisé pour l'implémentation des compilateurs
de langage fonctionnel.

L'inférence de type d'un langage de programmation fonctionnel est fortement liée à la théorie du $\lambda{}-calcul$.
En effet, le type d'unification utilisé dans ces langages est un sous-ensemble d'un problème indécidable.
Par conséquent, lors de la création du langage, il faut faire attention à ne pas sortir de ce sous-ensemble.
De plus, la représentation interne utilisée par le compilateur influe beaucoup sur les algorithmes
utilisés par l'inférence de type.

Le langage \typer{} et son compilateur étant en cours de développement, les
seul programmes écrit dans ce langage servent à tester le compilateur.
La création de programmes de moyenne ou grande taille permettra d'évaluer les avantages et les inconvénients de langages
sécurisé de la flexibilité pour cette sécurité.
