\chapter{Gestion de Projet}

\section{Planning}

\structpar{planning}
Le stage a été découpé en quatre parties.

\begin{figure}[!h]
    \centering
    \caption{Planning du stage}
\label{tab:plan}
    \begin{tabular}{|l | l | l |}
        \toprule
        partie & durée & résumé \\
        \midrule
        première partie & une semaine & familiarisation avec le code et les outils utilisés\\
        \midrule
        deuxième partie & quatre semaines & Implémentation de l'unification de premier ordre\\
        \midrule
        troisième partie & trois semaines & Implémentation de l'inversion des substitutions d'index de \textit{de Bruijn}\\
        \midrule
        quatrième partie & quatre semaines & Intégration de l'unification dans l'inférence de type et résolution des bugs\\
        \bottomrule
    \end{tabular}
\end{figure}

Les deux dernières semaines ont été laissées libres pour être utilisées pour résoudre les problèmes survenus lors de
l'implémentation des différentes parties ou au cas où une des parties du planning durerait plus longtemps que prévu.
\structpar{gantt}
\begin{figure}[!h]
    \centering
    \includegraphics[width=1\linewidth]{./plan-1.png}
    \caption{Première partie du diagramme de Gantt}
\label{fig:gant1}
\end{figure}
\begin{figure}[!h]
    \centering
    \includegraphics[width=1\linewidth]{./plan-2.png}
    \caption{Deuxième partie du diagramme de Gantt}
\label{fig:gant2}
\end{figure}
\structpar{rythmes des contacts}
Le rythme des contacts avec mon tuteur de stage était d'une fois par semaine. Ces contacts servaient à
faire un compte rendu de mon avancement durant cette semaine.
De plus, je pouvais contacter mon tuteur en cas de besoin, pour résoudre un problème théorique par exemple.

\section{Cahier des charges}

\structpar{cahier des charges}
Le but de ce stage était d'implémenter une première version d'inférence de type pour le compilateur
du langage \typer{}.

\structpar{définition de l'inférence de type}
L'inférence de type est une technique permettant à un compilateur de déduire le type d'une expression.
L'inférence de type de \typer{} doit permettre au système de type de fournir un typage fort.

L'inférence de type de \typer{} est proche de l'inférence de type de \textit{Hindley-Milner} et doit permettre
d'écrire du code sans aucune annotation de type.

\structpar{contraintes techniques}
Le code du compilateur de \typer{} est hébergé sur un dépôt gitlab.
L'implémentation de l'inférence de type devait être codée dans le même langage que le compilateur: \texttt{ocaml}.
Les outils de compilation et débogage étaient ceux fournis avec le compilateur \texttt{ocaml}: \texttt{ocamlbuild} et \texttt{make}
pour la compilation et \texttt{ocamldebug} pour le débogage.

Chaque partie du code devait être testée grâce à un framework de test créé pour le compilateur.
